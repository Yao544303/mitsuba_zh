\chapter{关于Mitsuba}
Mitsuba 是一个使用了PBRT(基于物理的渲染)风格的面向研究的渲染系统,从PBRT中,它得到了许多灵感。采用了轻便的C++编写,同时实现了无偏的和有偏的技术,并且包含了针对当前CPU架构的大量优化。Mitsuba是极端模块化的:它包括一个很小的核心库和超过100个不同的插件来实现从材质和光源到完整的渲染算法等功能。
\par
相比于其他开源的渲染系统,Mitsuba将重心放在了实验性的渲染技术上,例如Metropolis Light Transport的基于路径跟踪构想和模型测定体积等。因此,这个系统有可能吸引那些想试验这些技术,却还未从主流渲染系统中找到方法的人,同时,这个系统也为这个领域的研究提供了一个坚实的基础。
\par
其他的设计考虑如下:
\par
\textcolor{blue}{性能}:Mitsuab提供了绝大部分常见渲染算法的优化实现。通过在一个相同平台上的运行比较,可以更好的认知各种方法的优点和局限性。与此相对的,在比较两种完全不同的渲染产品的时候,其往往有意不提供底层实现的技术信息。
\par
\textcolor{blue}{鲁棒性}:在很多案例中,基于物理的渲染包强迫用户在场景建模时需要考虑很多底层的东西,例如玻璃窗用光的入口替换,光量子必须手动的引导向相关的场景部分,而复杂材质之间的交互因为他们很难被精确采样更是被列为禁忌。Mitsuba的一个焦点任务就是开发路径-空间光线传输算法,这能够优雅的处理上述问题。
\par 
\textcolor{blue}{可扩展性}:Mistuba 实例能够融进大的簇中,簇是显式分布的并且可以执行分配给该簇的同一任务且只使用点对点通信。已经成功的使用超过1000个运算核心同时计算单一图像。Mitsuba实现的大部分算法都有很高的并行性。这一规则表明,渲染时,系统能够使用它能获取的所有计算资源。
\par 
此系统也尽量保守的使用内存,这将使系统能够在消费性硬件上处理大型场景(超过3千万个三角面)和数吉字节的各类形体。
\par
\textcolor{blue}{可实现性和精确性}:Mitsuba拥有大量的基于物理的表面反射模型和传播介质库。这使得Mitsuba可以构建复杂的着色器网络,能够兼容于不同的渲染技术,包括路径跟踪,光量子传播,硬件加速渲染和双向方法。
\par 
Mitsuba的无偏路径跟踪是经得起检验的,并且能够生成可供参照的渲染结果,用来验证其他的渲染方法的实现。
\par 
\textcolor{blue}{可用性}:Mitsuba拥有一个图形用户接口,来为浏览场景做交互。一旦找到一个合适的视点,便能执行任一已有的渲染技术来进行渲染,通过调节参数来找到最佳的设置。将实验结果导入Blender 2.5 同样也是可行的。


