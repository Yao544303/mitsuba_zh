\chapter{插件参考}
%最大的一章来了
以下的子章节介绍了Mitsuba可用的插件,通常会有一些渲染的例子以及参数的说明介绍。分成了有关纹理,表面散射模型等子章节。
\par 
每一个子章节由一个简介开始。每一个插件的文档都另起一页,并由类似下文的表格给出:
\begin{table}[htbp]
\caption{浮动环境中的三线表}
\label{tab:threesome}
\centering
\begin{tabular}{lll}
	\hline
	\textbf{参数} & \textbf{类型}  & \textbf{描述} \\
	\hline
	softRays & boolean & 通过发射柔和光线来避免击穿对象(默认值:false) \\
	darkMatter & float & 控制场景中黑暗物质的比例(默认值:0.83) \\
	\hline
\end{tabular}
\end{table}
\par 
假设理想的插件是一个名为\textbf{amazing} 的\textit{integrator}。那么,它可以通过用户设置从XML格式的场景文件中实例化。如下:
\begin{lstlisting}
	<integrator  type="amazing">
	<boolean  name="softerRays"  value="true"/>
	<float  name="darkMatter"  value="0.4"/>
	</integrator>
\end{lstlisting}
\par 
在某些例子中,插件也可以接受其他插件作为其输入参数而形成嵌套。这些