\chapter{局限性}
Mitsuba 能够用来解决很多有趣的光线传输问题。但是,使用者需要意识到这个系统也存在如下固有局限性:
\par 
(i) 波动光学:Mitsuba是基于几何光学的,这也意味着这个系统无法模拟光的波动现象比如衍射。
\par 
(ii)偏振:Mitsuba 不支持光的偏振。换句话说,即假设光是随机偏振的。
\par 
(iii)数字精度:Mitsuba产生的任何结果的精度都是由底层浮点数指针运算估算的。
\par 
举个栗子,一个复杂的场景能够成功渲染出来,但是当场景中的物体从原始位置转移到较远距离的位置后,则可能出错。原因是在新的位置,浮点数指针变稀疏了。为了避免这种问题,需要对IEEE-754有一个很好理解。(废话 我的理解是 可能就8位, 原始是1111.1111 现在是11111111 木有小数点了。。)
